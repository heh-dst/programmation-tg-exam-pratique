% !TeX root = questionnaire.tex
\begin{tikzpicture}[remember picture, overlay]
	\node[anchor=south west, xshift=-3mm, yshift=0.5cm] at (0,0) {\includegraphics[height=48pt]{Logo-HEH-DST.png}};
\end{tikzpicture}%
\textbf{\thetitle{} \hfill{} \thesubtitle{}}

\vspace{1em}

\pagetitle{Consignes}

\begin{itemize}
	\item \textbf{Ne tournez pas cette page avant d'en avoir reçu l'autorisation.}
	\item Durée maximale~: \qty{60}{\minute} (\qty{45}{\minute} + \qty{15}{\minute} pour PAI).
	\item Chaque réponse correcte vaut 1~point. Il n'y a pas de point négatif.
	\item À chaque question correspond une (et une seule) réponse correcte.
	\item En plus des réponses spécifiques, des réponses générales font appel à votre vigilance.
	      Ces réponses générales sont~:
	      \begin{itemize}[noitemsep]
		      \item \textbf{aucune} si aucune réponse spécifique n'est correcte.
		      \item \textbf{toutes} si toutes les réponses spécifiques sont correctes.
		      \item \textbf{manque} s'il manque au moins une information nécessaire dans l'énoncé de la question.
		      \item \textbf{absurdité} si une absurdité s'est glissée dans l'énoncé de la question.
	      \end{itemize}
	      La réponse absurdité a priorité sur les autres réponses.
	\item Seule la feuille de réponse sera corrigée.
	      Vous pouvez donc écrire au recto et au verso des feuilles de question et de brouillon.
\end{itemize}
