\documentclass[11pt,french,a4paper,twoside]{scrartcl}

\usepackage{amsmath,booktabs,fontspec,graphicx,minted,polyglossia,setspace,siunitx,titling,tikz,xcolor}

\date{2025-01-12} % chktex 8
\title{Correction examen pratique}
\newcommand{\thesubtitle}{Programmation -- Bachelier en Techniques graphiques} % chktex 8
\author{François ROLAND}

\usepackage[sfdefault]{roboto}
\usepackage{roboto-mono}
\usepackage[mathrm=sym]{unicode-math}
\setmathfont{Fira Math}

\sisetup{
	locale=FR,
	per-mode=fraction
}

\setminted{
  autogobble=true,
  breaklines=true,
  fontsize=\footnotesize,
  linenos=true,
  stepnumber=5,
  tabsize=2
}

\setdefaultlanguage{french}

\usepackage{csquotes}

\usepackage{hyperref}
\hypersetup{
	pdftitle={\thetitle},
	pdfauthor={\theauthor},
	pdflang={fr-BE},
	hidelinks}

\onehalfspacing{}

\begin{document}
\noindent{}%
\begin{tikzpicture}[remember picture, overlay]
	\node[anchor=south west, xshift=-4.4mm, yshift=0.5cm] at (0,0) {\includegraphics[height=48pt]{Logo-HEH-DST.png}};
\end{tikzpicture}%
\noindent{}%
\noindent{}%
\textbf{\thetitle{} \hfill{} \thesubtitle{}}

\vspace{1em}

\section{Tests fonctionnels}

Ces tests doivent permettre de vérifier le bon fonctionnement du programme de l'étudiant.

\begin{enumerate}
	\item Demandez à l'étudiant de compiler son code et vérifiez que le programme démarre  pour vous assurez que le code exécuté est bien le code affiché dans l'éditeur.
	\item Vérifiez les résultats pour un triangle dont le cercle circonscrit a un rayon de \num{1}.
	      \begin{description}
		      \item[Sommets] \(\left(\num{1.00}, \num{0.00}\right)\); \(\left(\num{-0.50}, \num{0.87}\right)\); \(\left(\num{-0.50}, \num{-0.87}\right)\)
		      \item[Surface] \(\num{1.30}\)
	      \end{description}
	\item Vérifiez les résultats pour un carré dont le cercle circonscrit a un rayon de \num{1.5}.
	      \begin{description}
		      \item[Sommets] \(\left(\num{1.50}, \num{0.00}\right)\); \(\left(\num{-0.00}, \num{1.50}\right)\); \(\left(\num{-1.50}, \num{-0.00}\right)\); \(\left(\num{0.00}, \num{-1.50}\right)\)
		      \item[Surface] \(\num{4.50}\)
	      \end{description}
	\item Vérifiez les résultats pour un pentagone dont le cercle circonscrit a un rayon de \num{2}.
	      \begin{description}
		      \item[Sommets] \(\left(\num{2.00}, \num{0.00}\right)\); \(\left(\num{0.62}, \num{1.90}\right)\); \(\left(\num{-1.62}, \num{1.18}\right)\); \(\left(\num{-1.62}, \num{-1.18}\right)\); \(\left(\num{0.62}, \num{-1.90}\right)\)
		      \item[Surface] \(\num{9.51}\)
	      \end{description}
	\item Vérifiez les résultats pour un polygone à 12 côtés dont le cercle circonscrit a un rayon de \num{3}.
	      \begin{description}
		      \item[Sommets] \(\left(\num{3.00}, \num{0.00}\right)\); \(\left(\num{2.60}, \num{1.50}\right)\); \(\left(\num{1.50}, \num{2.60}\right)\); \(\left(\num{-0.00}, \num{3.00}\right)\); \(\left(\num{-1.50}, \num{2.60}\right)\); \(\left(\num{-2.60}, \num{1.50}\right)\); \(\left(\num{-3.00}, \num{-0.00}\right)\); \(\left(\num{-2.60}, \num{-1.50}\right)\); \(\left(\num{-1.50}, \num{-2.60}\right)\); \(\left(\num{0.00}, \num{-3.00}\right)\); \(\left(\num{1.50}, \num{-2.60}\right)\); \(\left(\num{2.60}, \num{-1.50}\right)\)
		      \item[Surface] \(\num{27.00}\)
	      \end{description}
	\item Essayez de saisir les valeurs suivantes pour le nombre de côtés et vérifiez que le programme les rejette avec un message d'erreur~: \num{-2}; \num{2}; \textquote{abc}; \num{3.1}.
	\item Essayez de saisir les valeurs suivantes pour le rayon du cercle et vérifiez que le programme les rejette avec un message d'erreur~: \num{-2}; \num{0}; \textquote{abc}.
	\item Vérifiez que le menu fonctionne permet de calculer un nouveau polygone et de quitter le programme.
\end{enumerate}

\section{Lecture du code}

\subsection{Fonctionnement}
\begin{enumerate}
	\item Vérifiez que le programme utilise bien un tableau pour stocker les coordonnées cartésiennes et les afficher.
\end{enumerate}

\subsection{Lisibilité}
\begin{enumerate}
	\item Indentation et cohérence des espaces et retours à la ligne
	\item Noms de variables et de fonctions
	\item Pertinence des commentaires
\end{enumerate}

\subsection{Design}
\begin{enumerate}
	\item Utilisation des structures de contrôle (boucles et conditions)
	\item Découpage en fonctions
	\item Structures de données
	\item Absence de code inutile
\end{enumerate}


\section{Résolu}

\begin{singlespace}
	\inputminted{C}{sources/polygons.c}
\end{singlespace}

\end{document}
