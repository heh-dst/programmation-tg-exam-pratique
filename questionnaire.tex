\documentclass[12pt,addpoints,french,a4paper,oneside]{exam}
% chktex-file 19

% !TeX root = questionnaire.tex
\usepackage{amsmath,booktabs,enumitem,fontspec,graphicx,multicol,polyglossia,setspace,siunitx,titling,tikz,xcolor}

\date{2025-01-12} % chktex 8
\title{Examen pratique}
\newcommand{\thesubtitle}{Programmation -- Bachelier en Techniques graphiques} % chktex 8
\author{François ROLAND}

\header{Date~: \fillin[][3cm]}{}{Nom~: \fillin[][8cm]}
\footer{}{\thepage{}}{}
\colorfillwithlines{}
\colorfillwithdottedlines{}
\setlength\linefillheight{.8cm}
\setlength\dottedlinefillheight{.8cm}

\usepackage[sfdefault]{roboto}
\usepackage{roboto-mono}
\usepackage[mathrm=sym]{unicode-math}
\setmathfont{Fira Math}

\sisetup{
	locale=FR,
	per-mode=fraction
}

\setdefaultlanguage{french}

\usepackage{csquotes}

\usepackage{hyperref}
\hypersetup{
	pdftitle={\thetitle},
	pdfauthor={\theauthor},
	pdflang={fr-BE},
	hidelinks}

\onehalfspacing{}

\usetikzlibrary{angles,calc,quotes}

\newcommand{\insertdraftpage}{}



\begin{document}
\pagestyle{foot}
% !TeX root = questionnaire.tex
\noindent{}%
\begin{tikzpicture}[remember picture, overlay]
	\node[anchor=south west, xshift=-4.4mm, yshift=0.5cm] at (0,0) {\includegraphics[height=48pt]{Logo-HEH-DST.png}};
\end{tikzpicture}%
\textbf{\thetitle{} \hfill{} \thesubtitle{}}

\vspace{1em}

\section*{Consignes}

\begin{itemize}
	\item \textbf{Ne tournez pas cette page avant d'en avoir reçu l'autorisation.}
	\item Durée maximale~: \qty{60}{\minute} (\qty{45}{\minute} + \qty{15}{\minute} pour PAI).
\end{itemize}



\section*{Polygones réguliers}

\subsection*{Contexte}

Il est possible de construire n'importe quel polygone régulier à l'aide d'un cercle.
Pour cela, il suffit de diviser le cercle en autant de parties égales que le polygone a de côtés.
Chaque sommet du polygone régulier est alors un point d'intersection entre le cercle et une droite passant par le centre du cercle.
Par exemple, pour un triangle équilatéral, on divise le cercle en trois parties égales.
L'angle au centre \(\theta{}\) vaut alors \(\frac{\ang{360}}{3} = \ang{120}\).
Pour un carré on a \(\frac{\ang{360}}{4} = \ang{90}\), pour un pentagone \(\frac{\ang{360}}{5} = \ang{72}\), et ainsi de suite.

\begin{center}
	\begin{tikzpicture}
		\coordinate (T) at (0,0);
		\path (T) +(0:2.5cm) coordinate (A) (T) +(120:2.5cm) coordinate (B);
		\path[help lines] pic["\footnotesize\ang{120}",draw,fill=gray!30,angle eccentricity=1.7,angle radius=0.4cm] {angle = A--T--B};
		\draw[help lines,dashed] (T) circle (2.5cm);
		\fill[help lines] (T) circle (2pt);
		\draw[help lines,dashed] foreach \angle in {0, 120, 240}{ (T) -- +(\angle:2.5cm) };
		\draw[thick] (T) ++(0:2.5cm)
		foreach \angle in {0, 120, 240}{ -- (\angle:2.5cm) }
		-- cycle;

		\coordinate (S) at (5.5,0);
		\path (S) +(0:2.5cm) coordinate (A) (S) +(90:2.5cm) coordinate (B);
		\path[help lines] pic["\footnotesize\ang{90}",draw,fill=gray!30,angle eccentricity=1.8,angle radius=0.4cm] {angle = A--S--B};
		\draw[help lines,dashed] (S) circle (2.5cm);
		\fill[help lines] (S) circle (2pt);
		\draw[help lines,dashed] foreach \angle in {0, 90, 180, 270}{ (S) -- +(\angle:2.5cm) };
		\draw[thick] (S) +(0:2.5cm)
		foreach \angle in {0, 90, 180, 270}{ -- +(\angle:2.5cm) }
		-- cycle;

		\coordinate (P) at (11,0);
		\path (P) +(0:2.5cm) coordinate (A) (P) +(72:2.5cm) coordinate (B);
		\path[help lines] pic["\footnotesize\ang{72}",draw,fill=gray!30,angle eccentricity=1.8,angle radius=0.4cm] {angle = A--P--B};
		\draw[help lines,dashed] (P) circle (2.5cm);
		\fill[help lines] (P) circle (2pt);
		\draw[help lines,dashed] foreach \angle in {0, 72, 144, 216, 288}{ (P) -- +(\angle:2.5cm) };
		\draw[thick] (P) +(0:2.5cm)
		foreach \angle in {0, 72, 144, 216, 288}{ -- +(\angle:2.5cm) }
		-- cycle;
	\end{tikzpicture}
\end{center}

Si \(n\) est le nombre de côtés du polygone, \(r\) le rayon du cercle et \(\theta{}\) l'angle au centre, la surface \(S\) d'un polygone régulier vaut~:
\begin{equation*}
	S = \frac{n r^2 \sin\left(\theta{}\right)}{2}
\end{equation*}

Rappelons également que les coordonnées cartésiennes \(\left(x, y\right)\) d'un point dont les coordonnées polaires sont \(\left(r, \theta{}\right)\)~ s'obtiennent à l'aide des formules suivantes~:
\begin{align*}
	x & = r \cos\left(\theta{}\right) \\
	y & = r \sin\left(\theta{}\right)
\end{align*}

\subsection*{Consignes}

\begin{enumerate}
	\item Vous devez créer un programme C qui permet de caluler les coordonnées cartésiennes de chaque sommet d'un polygone régulier ainsi que sa surface.
	\item Le programme doit demander à l'utilisateur le nombre de côtés du polygone et le rayon du cercle dès son démarrage.
	\item Une fois les données saisies, le programme doit~:
	      \begin{enumerate}
		      \item Calculer les coordonnées cartésiennes et les stocker dans un tableau.
		      \item Calculer la surface du polygone.
		      \item Afficher les coordonnées cartésiennes à partir du tableau et la surface.
	      \end{enumerate}
	\item Après l'affichage des données, le programme doit afficher un menu qui permet à l'utilisateur de~:
	      \begin{enumerate}
		      \item Modifier le nombre de côtés du polygone et le rayon du cercle.
		      \item Quitter le programme.
	      \end{enumerate}
	\item A tout moment, si l'utilisateur saisit une donnée incorrecte (par exemple, une lettre au lieu d'un nombre ou un nombre négatif), le programme doit afficher un message d'erreur et demander à l'utilisateur de saisir à nouveau la donnée.
	\item Le premier sommet du polygone doit être situé sur l'axe des \(x\).
	\item Le cercle doit être centré en \(\left(0, 0\right)\).
	\item Les valeurs réelles doivent être arrondies à 2 chiffres après la virgule.
\end{enumerate}

\subsection*{Barème d'évaluation}

\begin{questions}

	\titledquestion{Fonctionnement} Fonctionnement du code
	\begin{parts}
		\part[2] Saisie initiale
		\part[3] Menu
		\part[3] Calcul des coordonnées cartésiennes
		\part[2] Calcul de la surface
	\end{parts}

	\titledquestion{Lisibilité} Lisibilité du code
	\begin{parts}
		\part[2] Indentation et cohérence des espaces et retours à la ligne
		\part[2] Noms de variables et de fonctions
		\part[1] Pertinence des commentaires
	\end{parts}

	\titledquestion{Design} Design du code
	\begin{parts}
		\part[2] Utilisation des structures de contrôle (boucles et conditions)
		\part[1] Découpage en fonctions
		\part[1] Type de variable et structures de données
		\part[1] Absence de code inutile
	\end{parts}

\end{questions}

\newpage

\thispagestyle{headandfoot}

\section*{Évaluation de l'examen pratique}

Cette page est réservée à l'évaluateur. Merci de ne rien y écrire en dehors de la date et de votre nom.

\subsection*{Récapitulatif des points}

\begin{center}
	\gradetable[v][questions]
\end{center}

\subsection*{Commentaires}

\fillwithdottedlines{\stretch{1}}

\end{document}
